\documentclass[12pt,a4paper]{report}
\usepackage[utf8]{inputenc}
\usepackage[spanish]{babel}
\usepackage{amsmath}
\usepackage{amsfonts}
\usepackage{amssymb}
\usepackage{lmodern}
\usepackage{amsmath}
\usepackage{enumerate}
\usepackage[left=2cm,right=2cm,top=2cm,bottom=2cm]{geometry}
\usepackage{graphicx}
\usepackage{mathtools}
\usepackage{stackrel}
\renewcommand{\theequation}{\arabic{equation}}
\newcounter{neq}
\providecommand{\abs}[1]{\lvert#1\rvert}
\newcommand{\QED}{\hfill \textit{\textbf{Q.E.D.}}}
\author{Agustin Curto, agucurto95@gmail.com}
\title{Resumen de teorico para el final \\ de Probabilidad y Estadística}
\date{2015}


\begin{document}
\maketitle
\tableofcontents

\chapter{Estadstica Descriptiva}
	\par La estadística está dividida en dos grandes ramas:
	\begin{itemize}
		\item \textbf{Estadistica Descriptiva:} Organizar y resumir la información.
		\item \textbf{Estadistica Inferencial:} Extraer conclusiones acerca de hipótesis planteadas.
	\end{itemize}
	
	\vspace{3mm}
	\par \textbf{Población:} Colección de sujetos o elementos de interés, puede ser finita o infinita.
	
	\vspace{3mm}
	\par \textbf{Muestra:} Subconjunto de la población elegido al azar, de tamaño muestral \textit{n}.
	
	\vspace{3mm}
	\par \textbf{Tipos de datos de la muestra:}
	\begin{itemize}
		\item \textbf{Numéricos:} Discretos (determinados valores), Continuos (valores en un intervalo).
		\item \textbf{Categóricos:} Ordinal (orden), Nominal (no orden).
	\end{itemize}
	
	\vspace{3mm}
	\par \textbf{Intervalos de Clase:} Datos $\lbrace x_{1} \leq x_{2} \dotsc x_{n} \rbrace \subset \left[a, b \right]$, con $a < b$.
	\[
		\underbrace{\left[ \dotsc \right]}_{I_{1}} \underbrace{\left( \dotsc \right]}_{I_{2}} \underbrace{\left( \dotsc \right]}_{I_{3}} \underbrace{\left( \dotsc \right]}_{I_{4}}
	\]
	
	\vspace{3mm}
	\par \textbf{Tabla de Distribución de Frecuencia}
	\begin{table}[htbp]
		\begin{center}
			\begin{tabular}{|l|l|l|}
				\hline
				Iintervalos de Clase & Frecuencia Absoluta & Frecuencia Relativa \\ \hline
				$I_{1}$ & 3 & 3/8 \\ \hline
				$I_{2}$ & + 2 & + 2/8 \\ \hline
				$I_{3}$ & + 1 & + 1/8 \\ \hline
				$I_{4}$ & + 2 & + 2/8 \\ \hline
				Total & 8 & 1 \\
				\hline
			\end{tabular}
		\end{center}
	\end{table}

	\underline{Referencias:}
	\begin{itemize}
		\item MC: Marca de clase. Es el punto medio del intervalo.
		\item FAA: Frecuencia absoluta acumulada.
		\item FRA: Frecuencia relativa acumulada.
	\end{itemize}
	
	\vspace{3mm}
	\par \textbf{Histogramas:} Es el gráfico de mayor difusión y la representación gráfica de la distribución de frecuencia.
	
	\par \underline{Como hacerlo:}
	\begin{itemize}
		\item En una recta horizontal marcar los \textit{k} intervalos. ($k = \sqrt{n}$).
		\item En cada intervalo trazar un rectángulo cuya área sea proposicional al número de observaciones en el mismo.
		\item Altura de los triángulos: $h_{i} = \frac{FR_{i}}{long(IC_{i})}$
		\item Suma de las áreas de los rectángulos igual a 1.
	\end{itemize}
	
	\vspace{3mm}
	\par \textbf{Media muestral:} Es el promedio de los $x_{1}, x_{2} \dotsc x_{n}$ puntos, denotada $\overline{x}$:
	\[
		\overline{x} = \frac{\sum_{i=1}^{n} x_{i}}{n}
	\]
	
	\vspace{3mm}
	\par \textbf{Mediana muestral:} Es el valor medio de las observaciones:
	\begin{equation*}
		\widetilde{x} =
	  	\left \lbrace
	  	\begin{array}{l}
			n \textup{ impar} \rightarrow \frac{n + 1}{2} \\	
	    	n \textup{ par} \rightarrow \textup{promedio} \left(\frac{n}{2}, \frac{n}{2} + 1 \right)	
  		\end{array}
  		\right.
	\end{equation*}
	
	\vspace{3mm}
	\par \textbf{Primer Cuartil ($Q_{1}$):} Es la mediana de la primera mitad de las observaciones.
	
	\vspace{3mm}
	\par \textbf{Tercer Cuartil ($Q_{3}$):} Es la mediana de la segunda mitad de las observaciones.
	
	\vspace{3mm}
	\par \textbf{Media Recortada:} Es un termino medio entre $\overline{x}$ y $\widetilde{x}$. Una media recortada 10\% sería quitar el 10\% más pequeño y el 10\% más grande de la muestra para luego calcular la media.
	
	\vspace{3mm}
	\par \textbf{Rango:} Direfencia entre la máxima y la mínima ($x_{n} - x_{1}$).
	
	\vspace{3mm}
	\par \textbf{Rango Intercuartil (RIC):} Direfencia entre el tercer y primer cuartil ($Q_{3} - Q_{1}$).
	
	\vspace{3mm}
	\par \textbf{Varianza Muestral:}
	\[
		\sigma^{2} = S^{2} = \frac{\sum_{i=1}^{n} {(x_{i} - \overline{x})}^{2}}{n - 1}
	\]
	
	\vspace{3mm}
	\par \textbf{Desviación Estándar Muestral:}
	\[
		\sigma = S = \sqrt{\frac{\sum_{i=1}^{n} {(x_{i} - \overline{x})}^{2}}{n - 1}}
	\]
	
	\vspace{3mm}
	\par \textbf{Fórmula para calcular $S^{2}$:}
	\[
		S^{2} = \sum_{i=1}^{n} x_{i}^{2} - \frac{{(\sum_{i=1}^{n} x_{i})}^{2}}{n}
	\]
	
	\vspace{3mm}
	\par \textbf{Proposición:} Sea $x_{1}, x_{2} \dotsc x_{n}$ una muestra y c cualquier constante distinta de cero, entonces:
	\begin{itemize}
		\item Si $y_{1} = x_{1} + c$, $y_{2} = x_{2} + c \dotsc$ $y_{n} = x_{n} + c \Rightarrow S_{y}^{2} = S_{x}^{2}$ 
		\item Si $y_{1} = c \; x_{1}$, $y_{2} = c \; x_{2} \dotsc$ $y_{n} = c \; x_{n} \Rightarrow S_{y} = \lvert c \rvert \; S_{x}$ 
	\end{itemize}
	
	\vspace{3mm}
	\par \textbf{Coeficiente de Variación:} $\qquad \frac{S}{\overline{x}} \; 100\%$
	
	\vspace{3mm}
	\par \textbf{Diagramas de caja (Box Plot):}
	\begin{enumerate}
		\item Ordenar los datos de menor a mayor.
		\item Calcular $\widetilde{x}, Q_{1}, Q_{3}$ y RIC.
		\item Sobre un eje vertical u horizontal marcar los valores extremos ($x_{1}, x_{n}$) y los cuartiles ($Q_{1}$ y $Q_{3}$).
		\item Sobre este eje dibujar una caja, cuyo borde izquierdo sea $Q_{1}$ y el derecho $Q_{3}$.
		\item Dentro de la caja trazar una línea sobre la mediana.
		\item Trazar segmentos desde cada extremo de la caja hasta las observaciones más alejadas que no superen 1,5 RIC de los brodes correspondientes.
		\item Marcar con circunferencias aquellos puntos comprendidos entre 1,5 RIC y 3 RIC respecto del borde más cercano, esos son los puntos llamados \textit{anómalos suaves} y con asteríscos aquellos puntos que superen 3 RIC, estos puntos son llamados puntos \textit{anómalos extremos}.
	\end{enumerate}
	
	\begin{center}
    	\includegraphics[width=12cm, height=3cm]{./graphics/box_plot.png}
	\end{center}
	

\chapter{Probabilidad}
	\section{Modelo Probabilístico}
		\par \textbf{Espacio Muestral:} Es el conjunto de todos los resultados posibles del experimento. Se lo denota S.
		\vspace{3mm}
		\par \textbf{Evento:} Es cualquier subconjunto de S. Si el evento tiene un solo elemento se llama evento simple, si no, es un evento compuesto.
		\vspace{3mm}
		\par \textbf{Definición:} Cuando A y B (eventos) no tienen resultados en común, se dice que son eventos mutuamente excluyentes o disjuntos. Además $P(A \cap B) = \emptyset$.
		\vspace{3mm}
		\par A se dice \textbf{familia de eventos} si:
		\begin{itemize}
			\item $S \in A$
			\item Si $a \in A \Rightarrow \overline{a} \in A$
			\item $\lbrace a_{i} \rbrace_{i=1}^{\infty}$ tal que $a_{i} \in A \Rightarrow \cup_{i=1}^{\infty} a_{i} \in A$
		\end{itemize}
		
		\vspace{3mm}
		\par \textbf{Medida de Probabilidad:} Diremos que $P: A \rightarrow \left[0, 1\right]$, con A evento, es medida de probabilidad si:
		\begin{itemize}
			\item $0 \leq P(a) \leq 1 \;\; \forall a \in A$
			\item $P(S) = 1$
			\item $\lbrace a_{i} \rbrace_{i=1}^{\infty}$ con $a_{i} \in A \; \forall i$ y mutuamente disjuntos.	
		\end{itemize}
		
		\vspace{3mm}
		\par \textbf{Modelo Probabilístico:} Es una terna compuesta (S, A, P), espacio muestral, familia de eventos y medida de probabilidad respectivamente.
		
		\vspace{3mm}
		\par \textbf{Propiedades:} Dado un experimento, tenemos (S, A, P) entonces se puede probar que:
		\begin{itemize}
			\item Si $A \subset B \Rightarrow P(B - A) = P(B) - P(A) \qquad$ y $\qquad P(B) \geq P(A)$
			\item $P(\overline{A}) = 1 - P(A) \qquad\qquad P(\emptyset) = 0$
			\item $P(A \cup B) = P(A) + P(B) - P(A \cap B)$
		\end{itemize}
		
	\section{Técnicas de Conteo}
		\par \textbf{Reglas del producto:} Suponga que un conjunto consiste en colecciones ordenadas de \textit{k} elementos y que hay $n_{1}$ opciones posibles para el primer elemento; para cada elección del primer elemento, hay $n_{2}$ elecciones posibles del segundo elemento; $\dotsc$ para elección posible de los $k - 1$ elementos, hay $n_{k}$ elecciones del \textit{k-esimo} elemento. Entonces hay $n_{1} n_{2} \dotsc n_{k}$ \textit{k-tuplas} posibles.
		
		\vspace{3mm}
		\par \textbf{Definición:} Para cualquier secuencia ordenada de \textit{k} objetos tomada de un conjunto de \textit{n} objetos, el número de \textit{permutaciones} de tamaño \textit{k} que se pueden construir a partir de \textit{n} objetos, se denota $P_{k, n}$ y se define:
		\[
			P_{k, n} = \frac{n!}{(n-k)!}
		\]
		
		\vspace{3mm}
		\par \textbf{Definición:} Dado un conjunto de \textit{n} elementos distintos, cualquier subconjuto no ordenado de tamaño \textit{k} de los objetos, se llama \textit{combinación}. El número de combinaciones de tamaño \textit{k} que se puede formar a partir de \textit{n} objetos, se denota ${n \choose k} $, y se define:
		\[
			C_{n, k} = {n \choose k}  = \frac{n!}{k! (n-k)!}
		\]
		
		
	\section{Probabilidad Condicional}
		\par Sean A y B eventos tal que $P(B) > 0$, llamamos \textbf{probabilidad condicional} de A dado B, y denotamos $P(A | B)$, porbabilidad de A dado que ocurrio B, al evento:
		\[
			P(A | B) = \frac{P(A \cap B)}{P(B)}
		\]
		
		\vspace{3mm}
		\par \textbf{Definición:} Diremos que A y B son \textit{eventos independientes}, si:
		\[
			P(A \cap B) = P(A) \; P(B) \qquad\qquad \textup{con A, B} \in \textbf{a}
		\]
	
		\vspace{3mm}
		\par \textbf{Proposición:} Sean A y B eventos:
		\begin{itemize}
			\item $\overline{A}$ y $\overline{B}$ son independientes.
			\item $\overline{A}$ y B son independientes.
			\item A y $\overline{B}$ son independientes.
		\end{itemize}
		
		\vspace{3mm}
		\par \textbf{Definición:} Diremos que $A_{1}, A_{2} \dotsc A_{n}$ son \textit{mutuamente independientes} si $\forall I \in \lbrace 1, 2, \dotsc n \rbrace$ resulta que:
		\[
			P(\cap_{i, j \in I} A_{i, j}) = \prod_{i, j \in I} P(A_{i, j})
		\]
		
		\vspace{3mm}
		\par \textbf{Ley de la multiplicación:} Sean $\lbrace A_{i} \rbrace_{i=1}^{\infty}$
eventos en \textbf{a}, entonces:
		\[
			P(\cap_{i = 1}^{n} A_{i}) = P(A_{1}) \; P(A_{2} | A_{1}) P(A_{3} | A_{1} \cap A_{2}) \dotsc P(A_{n} | \cap_{i=1}^{n - 1} A_{i})
		\]
		
		\vspace{3mm}
		\par \textbf{Ley de Probabilidad Total:} Si $\lbrace A_{i} \rbrace_{i=1}^{\infty}$ son eventos disjuntos en \textbf{a} tal que $S = \cup_{i=1}^{n} A_{i}$ entonces $\forall B \in \textbf{a}$:
		\[
			P(B) = \sum_{i=1}^{n} P(A_{i}) \; P(B | A_{i})
		\]
		
		\vspace{3mm}
		\par \textbf{Teorema de Bayes:} Si $\lbrace A_{i} \rbrace_{i=1}^{\infty}$ son eventos disjuntos en \textbf{a} y $P(A_{i}) > 0 \; \forall i$ tal que $S = \cup_{i=1}^{n} A_{i}$, entonces para cualquier otro evento B tal que $P(B) > 0$:
		\[
			P(A_{i} | B) = \frac{P(A_{i} \cap B)}{P(B)} = \frac{P(A_{i}) \; P(B | A_{i})}{P(B)}
		\]
		donde $P(B) = \sum_{i=1}^{n} P(A_{i}) \; P(B | A_{i})$ como se dijo en la ley de probabilidad total.
	
	
\chapter{Variables Aleatorias Discretas}
\chapter{Variables Aleatorias Continuas}
\chapter{Distribución de Probabilidad Conjunta}
\chapter{Estimación Puntual}
\chapter{Intervalos de Confianza}
\chapter{Prueba de Hipótesis}



\begin{thebibliography}{X}
\bibitem{Baz} \textsc{Agustín Curto},
<<Carpeta de Clase, 2015>>,
\textit{FaMAF, UNC}.
\end{thebibliography}

\vspace{\fill}
\begin{center}
Por favor, mejorá este documento en github
\includegraphics[width=1cm]{graphics/github.png} \\
https://github.com/ResumenesFaMAF/resumenProbYEst
\end{center}
\end{document}
